\documentclass[11pt,spanish]{article}

\usepackage{babel}
\usepackage[utf8]{inputenc}
\usepackage[T1]{fontenc}

\usepackage{palatino}
\usepackage[pdftex]{hyperref}
\usepackage{pgf}
\usepackage[final]{svninfo}
\usepackage[]{a4wide}
\usepackage{amsmath}
\usepackage{amsfonts}
\usepackage{graphicx}

%Definitions of numbers
\def\Nset{\mathbb{N}}
\def\Rset{\mathbb{R}}

\newcommand{\en}{\mbox{\ en \ }}
\newcommand{\sobre}{\mbox{\ sobre \ }}

\title{Diferencias finitas con Python}
\author{J. Rafael Rodríguez Galván \\[2em] \includegraphics[width=5em]{cc-by-sa}}

\begin{document}

\maketitle

  En este documento utilizaremos Python para  la
  aproximación numérica de ecuaciones en derivadas parciales mediante
  el método de las diferencias finitas 1D y 2D.

El método de las diferenciasfinitas consiste en reemplazar cada una de
las derivadas parciales por aproximaciones mediante cocientes
incrementales de orden uno o dos. Entre sus ventajas, se encuentra el
tratarse de un método intuitivo y fácil de implementar. Su principal
inconveniente: no es sencilla su implementación en dominios que no sean
de intervalos (1D), rectángulos (2D), cubos (3D).

\section{Diferencias finitas 1D}
Consideremos el siguiente problema diferencial de orden 2:
Calcular $u:[a,b] \to  \Rset$ solución de
\begin{align}
  \label{pb1d}
  -\frac{d^2 u(x) }{dx^2} = f(x) \en [a,b], \\
  u(a)=u_a, \ u(b)=u_b,
\end{align}
donde tanto $f$ como los datos de contorno, $u_a$ y $u_b$ son
previamente conocidos. Bajo unas hipótesis mínimas de regularidad para
$f$ (bastaría, por ejemplo, $f\in L^2(a,b)$, pero para utilizar el método de las
diferencias finitas supondremos que $f$ es contínua), se sabe que el
problema de contorno anterior tiene una única solución.

\subsection{Aproximación mediante el método de diferencias finitas}
\label{sec:diferencias-finitas-dim-1}
Consideremos una partición del intervalo $[a,b]$ en $n+1$
subintervalos, todos ellos de longitud $h=(b-a)/(n+1)$:

$$
a=x_0 < x_1 < \cdots < x_{n+1} = b.
$$

Calcularemos, en cada uno de los puntos $x_i$, una aproximación del
valor de $u(x_i)$, a la que llamaremos $u_i$. Conocemos los valores $u_0$ y
$u_{n+1}$ (pues deben coincidir, respectivamente, con los datos $u_a$
y $u_b$), de forma que nuestras incógnitas son, exactamente, $u_1,
u_2,\dots,u_n$. Para calcularlas, realizamos, en el sistema anterior,
una aproximación de segundo orden\footnote{Que no es difícil de
  deducir, utilizando los desarrollos de Taylor de $u$ en
$x-h$ y en $x+h$. Se puede demostrar, de hecho, que si $u \in
C^4([a,b])$, el esquema propuesto es  consistente
y convergente (de orden $2$ en $h$).} de la derivada segunda
:
$$
u''(x) \approx \frac{u(x-h)-2u(x)+u(x+h)}{h^2}.
$$

Utilizando esta fórmula en (\ref{pb1d}) para los puntos $x_i$
($i=1,\dots,n$), obtenemos el siguiente sistema de ecuaciones:
$$
-\frac{u_{i-1}-2u_i + u_{i+1}}{h^2} = f_i, \quad i=1,\dots,n.
$$
siendo $f_i=f(x_i)$.
No es difícil concluir que su expresión matricial es:

\begin{gather}
Au_h=f,\label{eq:sistema1D} \\
\intertext{para la matriz}
A=\frac{1}{h^2}\left(
\begin{array}{rrrrrr}
   2 & -1 &  0 &  0 & \dots & 0 \\
  -1 &  2 & -1 &  0 & \dots & 0 \\
   0 & -1 &  2 & -1 & \dots & 0 \\
     &    & \ddots & \ddots & \ddots \\
   0 & \dots & & -1 & 2  & -1 \\
   0 & \dots & & 0 & -1 & 2
 \end{array}
\right),
\\
\intertext{y los vectores}
u_h=
\begin{pmatrix}
  u_1 \\ u_2 \\ \vdots \\ u_{n-1} \\ u_n
\end{pmatrix}
%
\quad \mbox{y}  \quad
f=
\begin{pmatrix}
  f_1 + u_a/h^2 \\ f_2 \\ \vdots \\ f_{n-1} \\ f_n+u_b/h^2
\end{pmatrix}
\end{gather}
La matriz $A$ es tridiagonal y definida positiva, lo que dota al
sitema lineal annterior de muy buenas propiedades de cara a su
implementación y resolución, especialmente pensando en que $n$ sea muy
grande.

\subsection{Resultados numéricos con Octave}
Para construir el sistema (\ref{eq:sistema1D}), podemos utilizar la
función ``diag'' (ver la ayuda de Octave).

Se propone resolver la ecuación (\ref{pb1d}) para los siguientes
parámetros:
\begin{itemize}
\item $[a,b]=[0,1]$
\item $n= 10$;
\item $h = 1/(n+1)$;
\item Datos de contorno: $u_a=0$; $u_b=1$;
\item Segundo miembro: $f(x)=\frac{\pi^2}{4}  \sin(x\pi/2)$
\end{itemize}

\textbf{Extensión}: definir una función, \verb|diferencias_finitas_1d|,
que tome como parámetros:
\begin{itemize}
\item Los extremos del intervalo, $a$, $b$ y el tamaño de la
  partición, $n$.
\item Los datos de contorno, $u_a$ y $u_b$, y la función segundo
  miembro, $f$.
\end{itemize}
La función debe devolver el vector solución, $u$.

\section{Diferencias finitas 2D}

\subsection{Aproximación mediante el método de diferencias finitas}
\label{sec:diferencias-finitas-dim-2}

Consideremos un dominio rectangular, $\Omega=(a_x,b_x)\times(a_y,b_y)
\subset \Rset^2$.
Planteamos el problema: Calcular $u:\Omega \to  \Rset$ solución de
\begin{align}
  \label{pb1d}
  -\frac{\partial^2 u(x,y) }{\partial x^2}
  -\frac{\partial^2 u(x,y) }{\partial y^2}
  = f(x,y) \en \Omega, \\
  u=g \sobre \partial\Omega
\end{align}
donde $\partial\Omega$ es la frontera de $\Omega$ y tanto $f(x,y)$
como $g(x,y)$ son funciones conocidas, que supondremos continuas.


\subsection{Aproximación mediante el método de diferencias finitas}
\label{sec:diferencias-finitas-dim-2}

Definiremos un mallado uniforme del rectángulo $\Omega$, a través de
una partición de $[a_x,b_x]$ con talla $h_x = (b_x-a_x)/(N_x+1)$ y de
una partición de $[a_y,b_y]$ con talla $h_y = (b_y-a_h)/(N_y+1)$.

Estas particiones definen los puntos $(x_i,y_j)\in\Rset^2$, donde
\begin{align*}
  x_i &= i\cdot h_x, i=0,1,...,N_x+1
  \\
  y_j &= j\cdot h_y, j=0,1,...,N_y+1
\end{align*}

Calcularemos, en cada uno de los puntos $(x_i,y_j)$, una aproximación del
valor de $u(x_i,y_j)$, a la que llamaremos $u_{i,j}$.

De forma similar al caso $1D$, podemos aproximar cada una de las
derivadas parciales. Sumándolas, obtendremos la siguiente aproximación:
\begin{align*}
  \frac{\partial^2 u(x,y) }{\partial x^2}
  +
  \frac{\partial^2 u(x,y)}{\partial y^2}
  & \approx \frac{u(x-h_x,y)-2u(x,y)+u(x+h_x,y)}{h_x^2}
  \\
  &+
  \frac{u(x,y-h_y)-2u(x,y)+u(x,y+h_y)}{h_y^2}
\end{align*}

Utilizando esta fórmula en los puntos $(x_i,y_j)$ de la malla,
obtenemos el siguiente sistema de ecuaciones:
$$
-\frac{u_{i-1,j}-2u_{i,j} + u_{i+1,j}}{h_x^2}
-\frac{u_{i,j-1}-2u_{i,j} + u_{i,j+1}}{h_y^2}
= f_{i,j},
$$
siendo $f_{i,j}=f(x_{i},y_j), \  \quad i=1,\dots,N_x,  \quad i=1,\dots,N_y.$.

Para transformar las ecuaciones anteriores en un sistema lineal,
debemos escribir nuestras incógnitas, $u_{i,j}$ en forma de
vector. Para ello, basta realizar una simple renumeración de los
índices, por ejemplo, concatenando todas las filas, podemos definir el
siguiente vector:
$$
u=(u_{1,1}, u_{2,1},...,u_{N_x,1},\ u_{1,2}, u_{2,2},...,u_{N_x,2},\
...,\ u_{1,N_y}, u_{2,N_y},...,u_{N_x,N_y})
$$


Podemos comprobar que las ecuaciones anteriores se ajustan a la expresión
matricial $Au=b$ para:
$$
A=
\begin{pmatrix}
   a & -b &  0 &  0 & \dots & -c & 0 & \dots & 0 \\
  -b &  a & -b &  0 & \dots & 0 & -c & \dots & 0\\
   0 & -b &  a & -b & \dots & 0 & 0 & \ddots & 0 \\
   \vdots &    & \ddots & \ddots & \ddots & \\
   -c & 0 & \dots & -b & a  & -b \\
   0 & -c &  0 & \dots & -b & a  & -b \\
     & & &&    & \ddots & \ddots & \ddots & \\
   0 & \dots & &  & & & 0 & -b & a
 \end{pmatrix},
$$
siendo:
\begin{align*}
  a&=\frac{2}{h_x^2}+\frac{2}{h_y^2} \\
  b&=\frac{1}{h_x^2} \\
  c&=\frac{1}{h_y^2}
\end{align*}
El vector diagonal $(c,c,\dots,c)$ está separado exactamente $N_x$
posiciones de la diagonal de la matriz.

En cuanto al segundo miembro, incluirá a los valores
$f_{i,j}=f(x_i,y_j)$ (a través de una renumeración idéntica a la
realizada para el vector $u$) y a los términos frontera
$(1/h_x^2)g_{i,j}$ y $(1/h_y^2)g_{i,j}$, en las posiciones adecuadas.
Para simplificar el problema, podemos asumir que la condición de
contorno $g$, es nula salvo en el lado del rectángulo $y=y0$, es
decirq correspondiente a los puntos
$(x_0,y_0), (x_1,y_0),...,(x_{N_x+1},y_0)$.

En tal caso, debemos sumar a los $N_x$ primeros términos segundo
miembro los valores:
\begin{align*}
  \frac{1}{h_y^2} g_{i,0}, \quad i=1,...,N_x
\end{align*}


\subsection{Resultados numéricos con Python}
Se propone resolver la ecuación (\ref{pb1d}) para los siguientes
parámetros:
\begin{itemize}
\item $\Omega=[0,1]^2$
\item $N_x=N_y=10$;
\item $f=0$, $g=1$ en el lado $y=y_0=0$.
  $(x_0,y_0), (x_1,y_0),...,(x_{N_x+1},y_0)$. $f=g=0$ en el resto de
  la frontera.
\end{itemize}



\end{document}
%%% Local Variables:
%%% mode: latex
%%% TeX-master: t
%%% End:
